\documentclass{beamer}
\usetheme{Hannover}
\usetheme[orientation=landscape,size=custom,width=12.7,height=7.20,scale=0.5,debug]{Hannover}  %maps it to the screen orientation

%Title%
\title{R as a tool for digital survey data}
\subtitle{FOAR705 Proof of Concept presentation}
\author{Kathryn Phillips, 43686826} 
\institute{Macquarie University}
\date{\today}

%document%
\begin{document}
%content goes here%
%Title page%
\begin{frame}
\frame{\titlepage}
\end{frame}

%contents table%
\begin{frame}
\frametitle{Table of Contents}
\tableofcontents
\end{frame}

%section one -Introduction%%%%%%%
\section{Introduction}
%%%%%%%%%%%%%%%%%%%%%%%%%%%%%%%%%
\begin{frame}
\frametitle{Introduction}
%include font and spacing?%
My proof of concept for FOAR705 centred around constructing a tool chain using the data analysis software "R". As my MRes research topic is centred around examining the influence of Korean popular music (K-Pop) within the Japanese music market, I plan to collect large quantities of data on listening tastes and the recognition/interest in K-Pop from Japanese residents: especially within the Tokyo Region. 
\newline
\newline
 Manual collation of such data, or even the use of Excel spreadsheets could prove difficult and time-consuming. Therefore, I decided that attempting to create a simple tool chain, using R to collect and extrapolate CSV-formatted data sets and subsets, before producing easy-to-interpret visual data (in the form of graphs) would greatly reduce the amount of time spent on analysing the surveys in my actual research time frame.
 \end{frame}

%section 2 - Initial Scoping (subsection - selection)%%%%%%%
\section{Scoping}
\subsection{Selection}
%%%%%%%%%%%%%%%%%%%%%%%%%%%%%%%%%%%
\begin{frame}
\frametitle{Scoping}
\framesubtitle{Selection}
%font and spacing to be added%
Initially, I had several areas where using a digital tool could improve the flow and time-consumption of my work.
\newline
\item - Compiling sources for literature research
\item - Analysis of survey data
\item - Transcription and Translation of Interviews
\item - Writing and formatting my final Thesis
\newline
\newline
Out of these four areas, I considered survey data analysis to be the most time-consuming, so I decided to work on saving time here through implementing a digital tool.
\end{frame}

%%%%%%%%%%%%%%%%%%%%%%%%%%%%%%%%%%%%%%%
\subsection{Elaboration}
%%%%%%%%%%%%%%%%%%%%%%%%%%%%%%%%%%%%%%%
\begin{frame}
\frametitle{Scoping}
\framesubtitle{Elaboration}
In order to choose the correct tool to use, I had to refine my input and output sources. These went through many iterations, but eventually ended up like this:
\newline
\newline
- Input: Raw data collected from an online survey site that is capable of producing both qualitative and quantitative data and presenting it in a CSV format.
\newline
- Output: A dataset that can receive queries for specific subsets, answer these queries and plot the results in a graphical format.
\newline
\newline
Due to my choice of input and output, I was able to pick R as my intended tool-of-choice.
\end{frame}

%New Section - Implementation%%%%
\section{Implementation}
\subsection{Creating a Tool Chain}
%%%%%%%%%%%%%%%%%%%%%%%%%%%%%%%%%
\begin{frame}
\frametitle{Implementation}
\framesubtitle{Creating a Tool Chain}
To begin with, I needed to prove that R could take a basic .csv input and create my desired output. To do this, I began by creating my own small table of data, before loading it into the R environment and constructing a plot.  
\newline
\newline
This tool chain was a success, so I was then able to continue onto using more complex data, in a similar style to what I intended on using in my final thesis.
\end{frame}

%%%%%%%%%%%%%%%%%%%%%%%%%%%%%%%%%%%%%%%
\subsection{Elaborating on my Chain}
%%%%%%%%%%%%%%%%%%%%%%%%%%%%%%%%%%%%%%%
\begin{frame}
\frametitle{Implementation}
\framesubtitle{Elaborating on my Chain}
Now I knew that my tool chain worked with simple data, I decided to create a more complicated chain. This time, I would create a survey; mimicking question formats potentially suitable for my thesis.
\newline
For this, I created a simple survey via Google Forms. The number of questions were still kept to a minimum, and I asked my classmates and co-convenors to fill out this questionnaire. I then intended to construct normalised data grids within R, save them each as new, separate CSVs, before taking each and constructing graphs from them.
\newline
However, I have not yet completed this task. I intend to focus on this task in the next few weeks (until the date of the final formal presentation).
\end{frame}

%%%%%%%%%%%%%%%%%%%%%%%%%%%%%%%%%%%%%%%
\section{Results}
\subsection{Pros}
%%%%%%%%%%%%%%%%%%%%%%%%%%%%%%%%%%%%%%%
\begin{frame}
\frametitle{Results}
\framesubtitle{Pros:}
\item -General growth of my technical skillset
\item -Success in using R to create simple tool chains
\item -Increased knowledge of technical tools available for implementation in a range of tasks
\end{frame}

%%%%%%%%%%%%%%%%%%%%%%%%%%%%%%%%%%%%%%%
\subsection{Cons}
%%%%%%%%%%%%%%%%%%%%%%%%%%%%%%%%%%%%%%%
\begin{frame}
\frametitle{Results}
\framesubtitle{Cons:}
\item -R support online is often outdated for the current version (code changes)
\item -Time is still currently an issue in terms of utilising this tool - however, I would say that if R is implemented properly, there will likely be fewer problems concerning data breaking under manipulation. 
\end{frame}

%%%%%%%%%%%%%%%%%%%%%%%%%%%%%%%%%%%%%%%
\section{Conclusion}
%%%%%%%%%%%%%%%%%%%%%%%%%%%%%%%%%%%%%%%
\begin{frame}
\frametitle{Conclusion}
So far, my implementation of R as a tool for data analysis; taking simple raw csv data from online survey sites and producing clear graphical data for word processing and typesetting software has been successful. It still remains to be seen if I can effectively manipulate more complicated data. However, I am confident that with this technical skillset I have learnt in the process of undertaking my Proof of Concept, I will be more prepared to solve potential problems with my data, and will have saved time that will be valuable elsewhere during my MRes course.
\end{frame}

%done!%
\end{document}