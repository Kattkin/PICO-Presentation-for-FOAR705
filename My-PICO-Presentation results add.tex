\documentclass{beamer}
\usetheme{Hannover}
\usetheme[orientation=landscape,size=custom,width=12.7,height=7.20,scale=0.5,debug]{Hannover}  %maps it to the screen orientation
\usepackage{graphicx}
\graphicspath{{C:/Users/Katy/Documents/Uni/Postgrad/FOAR705/Week_12_presentation_PICO/Tries_again/}}
\usepackage{wrapfig}

%Title%
\title{Analysing Survey Data in R}
\subtitle{FOAR705 Proof of Concept presentation}
\author{Kathryn Phillips, 43686826} 
\institute{Department of International Studies}
\date{\today}

%document%
\begin{document}
%content goes here%
%Title page%
\begin{frame}
\frame{\titlepage}
\end{frame}

%contents table%
\begin{frame}
\frametitle{Table of Contents}
\tableofcontents
\end{frame}

%section one -Introduction%%%%%%%
\section{Introduction}
\subsection{My Research Topic}
%%%%%%%%%%%%%%%%%%%%%%%%%%%%%%%%%
\begin{frame}
\frametitle{My Research Topic}
During my MRes course, I plan to research the influence of Korean Pop (Kpop) within Japan.
\newline
You can find advertisements for Kpop all around Tokyo, but where is the research on this phenomenon?
\newline
\newline
\includegraphics[scale=0.2]{{bts}}
\includegraphics[scale=0.6]{{kpopexample}}
\centering

\end{frame}
%%%%%%%%%%%%%%%%%%%%%%%%%%%%%%%%%%
\subsection{Data Analysis in R}
%%%%%%%%%%%%%%%%%%%%%%%%%%%%%%%%%%

\begin{frame}
\frametitle{Data Analysis in R}
%include font and spacing?%
What happens if data is missing? 
\newline
What happens if you need to look at a specific section of the data you have collected?
\includegraphics[scale=0.5]{{why.jpg}}
\centering
\newline
You might end up wasting a lot of valuable time trying to solve these problems.
 \end{frame}

%section 2 - Initial Scoping (subsection - selection)%%%%%%%
\section{Scoping}
\subsection{Selection}
%%%%%%%%%%%%%%%%%%%%%%%%%%%%%%%%%%%
\begin{frame}
\frametitle{Scoping}
\framesubtitle{Selection}
%font and spacing to be added%
Initially, I had several areas where using a digital tool could improve the flow and pace of my work.
\newline
\item - Compiling sources for literature research
\item - Analysis of survey data
\item - Transcription and Translation of Interviews
\item - Writing and formatting my final Thesis
\newline
\newline
Out of these four areas, I considered survey data analysis to be the most time-consuming, so I decided to work on saving time here through implementing a digital tool.
\end{frame}

%%%%%%%%%%%%%%%%%%%%%%%%%%%%%%%%%%%%%%%
\subsection{Elaboration}
%%%%%%%%%%%%%%%%%%%%%%%%%%%%%%%%%%%%%%%
\begin{frame}
\frametitle{Scoping}
\framesubtitle{Elaboration}
In order to choose the correct tool to use, I had to refine my input and output sources. These went through many iterations, but eventually ended up like this:
\newline
\newline
- Input: Raw data collected from an online survey site that is capable of producing both qualitative and quantitative data and presenting it in a CSV format.
\newline
- Output: A dataset that can receive queries for specific subsets, answer these queries and plot the results in a graphical format.
\newline
\newline
Due to my choice of input and output, I was able to pick R as my intended tool-of-choice.
\end{frame}

%New Section - Implementation%%%%
\section{Implementation}
\subsection{Creating a Tool Chain}
%%%%%%%%%%%%%%%%%%%%%%%%%%%%%%%%%
\begin{frame}
\frametitle{Implementation}
\framesubtitle{Creating a Tool Chain}
To begin with, I needed to prove that R could take a basic .csv input and create my desired output. To do this, I created my own small table of data, before loading it into the R environment and constructing a plot.  
\newline
\newline
This tool chain was a success, so I was then able to continue onto using more complex data, in a similar style to what I intended on using in my final thesis.
\end{frame}

%%%%%%%%%%%%%%%%%%%%%%%%%%%%%%%%%%%%%%%
\subsection{Elaborating on my Chain}
%%%%%%%%%%%%%%%%%%%%%%%%%%%%%%%%%%%%%%%
\begin{frame}
\frametitle{Implementation}
\framesubtitle{Elaborating on my Chain}
Now I knew that my tool chain worked with simple data, created a more complicated chain. 
\begin{center}
Google Forms survey (real-world data collection) 
\newline
$\,\to\,$
Data analysis in R
\newline
$\,\to\,$ 
Graph imported into a Word document
\end{center}
\newline
The survey was completed by my FOAR705 class. With this new data I created a tabular database that could answer all the questions I asked.
\end{frame}

%%%%%%%%%%%%%%%%%%%%%%%%%%%%%%%%%%%%%%%
\section{Results}
\subsection{Pros}
%%%%%%%%%%%%%%%%%%%%%%%%%%%%%%%%%%%%%%%
\begin{frame}
\frametitle{Results}
\framesubtitle{Pros:}
\item -General growth of my technical skillset
\item -Success in using R to create simple tool chains
\item -Increased knowledge of technical tools available for implementation in a range of tasks
\begin{figure}[h]
\includegraphics[scale=0.3]{{Final}}
\end{figure}
\end{frame}

%%%%%%%%%%%%%%%%%%%%%%%%%%%%%%%%%%%%%%%
\subsection{Cons}
%%%%%%%%%%%%%%%%%%%%%%%%%%%%%%%%%%%%%%%
\begin{frame}
\frametitle{Results}
\framesubtitle{Cons:}
\item -R support online is often outdated for the current version (code changes)
\item -Time is still currently an issue in terms of utilising this tool - however, I would say that if R is implemented properly, there will likely be fewer problems concerning data breaking under manipulation. 
\end{frame}

%%%%%%%%%%%%%%%%%%%%%%%%%%%%%%%%%%%%%%%
\section{Conclusion}
\subsection{Final Words}
%%%%%%%%%%%%%%%%%%%%%%%%%%%%%%%%%%%%%%%
\begin{frame}
\frametitle{Conclusion}
My implementation of R as a tool for data analysis has been successful.
\newline 
I am now able to ask questions of R, and receive graphical responses with ease.
\newline 
This success means that I can replicate the same process shown here with my thesis data with minimal difficulty.
\end{frame}

%%%%%%%%%%%%%%%%%%%%%%%%%%%%%%%%%%%%%%%
\subsection{Code Repository}
%%%%%%%%%%%%%%%%%%%%%%%%%%%%%%%%%%%%%%%
\begin{frame}
\frametitle{Conclusion}
\framesubtitle{Code Repository}
If you would like to have a look at my code and the collected data used in this project, please follow the link below:
\newline
\newline
\url{https://github.com/Kattkin/FOAR705-Coding-line}
\end{frame}
 
%done!%
\end{document}